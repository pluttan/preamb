\documentclass[a4paper, 20pt,reqno]{article}
\usepackage[left=2cm,right=2cm,top=2cm,bottom=2cm]{geometry}
\usepackage[T2A]{fontenc}
% \DeclareMathSizes{10}{10}{10}{10}

\usepackage[russian]{babel}
\usepackage{soul}
\usepackage{mathtools}
\usepackage{gensymb}
\usepackage{amsfonts,amsmath,amssymb}
\usepackage{mathrsfs}
\usepackage{graphicx}
\usepackage[normalem]{ulem}
\usepackage[document]{ragged2e}
\usepackage{stmaryrd}
\usepackage{wrapfig}
\usepackage{fancyhdr}
\usepackage{floatflt}
\usepackage{python}
\usepackage{float}
\usepackage{amssymb}
\usepackage[most]{tcolorbox}
\usepackage{indentfirst}
\usepackage{setspace}
\usepackage{scrextend}
\usepackage{listings}
\usepackage{makecell,tabularx}
\usepackage{hyperref}
\usepackage{xcolor}

\renewcommand{\copyright}{pluttan}
\renewcommand{\mid}{|}
\newcommand{\docopyright}{\href{https://t.me/botva_its6}{$\mathfrak{Copyright}\ \mathfrak{botva}$}}
\newcommand{\rub}{{\rm{Р}\kern-.635em\rule[.5ex]{.52em}{.04em}\kern.11em}}

\definecolor{linkcolor}{HTML}{000000} 
\definecolor{urlcolor}{HTML}{0000FF} 
\definecolor{opasidecolor}{HTML}{FFF000}
\definecolor{hrfcolor}{HTML}{0000EE}
\newcommand{\hrf}[2]{{\color{hrfcolor}\href{#1}{#2}}}

\hypersetup{pdfstartview=FitH,pdfborder={0 0 0},  linkcolor=opasidecolor,urlcolor=urlcolor, colorlinks=false}

\definecolor{grey}{RGB}{40, 40, 40}

\definecolor{codegreen}{rgb}{0,0.6,0}
\definecolor{codegray}{rgb}{0.5,0.5,0.5}
\definecolor{codepurple}{rgb}{0.7,0,0.82}
\definecolor{mermaid}{rgb}{0.5,0.3,1}
\definecolor{backcolour}{rgb}{0.95,0.95,0.92}

\lstdefinestyle{mystyle}{
    backgroundcolor=\color{backcolour},   
    commentstyle=\color{codegreen},
    keywordstyle=\color{mermaid},
    numberstyle=\tiny\color{codegray},
    stringstyle=\color{codepurple},
    basicstyle=\ttfamily\footnotesize,
    breakatwhitespace=false,         
    breaklines=true,                 
    captionpos=b,                    
    keepspaces=true,                 
    numbers=left,                    
    numbersep=5pt,                  
    showspaces=false,                
    showstringspaces=false,
    showtabs=false,                  
    tabsize=2,
  extendedchars=true , % включаем не латиницу 
  escapechar=|, % |«выпадаем» в LATEX|
  frame=tb , % рамка сверху и снизу 
  commentstyle=\itshape , % шрифт для комментариев 
  stringstyle=\bfseries
}

\lstset{style=mystyle}

\lstdefinestyle{CommentStyle}{
    language=XML,
    %numbers=left, numberstyle=\tiny, stepnumber=1, numbersep=5pt,
    commentstyle=\color{red},
	basicstyle=\footnotesize\ttfamily,
	language={[ANSI]C++},
	keywordstyle=\bfseries,
	showstringspaces=false,
	morekeywords={include, printf},
	commentstyle={},
	escapeinside=§§,
	escapebegin=\begin{russian}\commentfont,
	escapeend=\end{russian},
    keywordstyle=\color{blue}\bfseries,
    morekeywords={align,begin},
    extendedchars=\true,
    tabsize=2
}
\lstdefinestyle{myLatexStyle}{
    language=c++,
    %backgroundcolor=\color{grey},
    numbers=left, numberstyle=\tiny, stepnumber=1, numbersep=5pt,
    commentstyle=\color{red},
    keywordstyle=\color{blue}\bfseries,
    morekeywords={align,begin},
    extendedchars=\true,
    tabsize=2
}

\lstdefinestyle{pmyLatexStyle}{
    language=java,
    %backgroundcolor=\color{grey},
    numbers=left, numberstyle=\tiny, stepnumber=1, numbersep=5pt,
    commentstyle=\color{red},
    keywordstyle=\color{blue}\bfseries,
    morekeywords={align,begin},
    extendedchars=\true,
    tabsize=2
}

\definecolor{block-gray}{gray}{0.90} % уровень прозрачности (1 - максимум)
\definecolor{yellow}{HTML}{F0FFFF}
\newtcolorbox{mainQuote}{colback=block-gray,grow to right by=-10mm,grow to left by=-10mm,boxrule=0pt,boxsep=0pt,breakable} % настройки области с изменённым фоном
\newtcolorbox{myquote2}{colback=yellow,grow to right by=-10mm,grow to left by=-10mm,boxrule=0pt,boxsep=0pt,breakable} % настройки области с изменённым фоном

\setlength{\parindent}{12,5mm}

\newcommand{\logo}{
    % \vcenter{\hbox{\includegraphics[width=.8em]{/root/BMSTU/forMyDocs/preamb/bw2.png}}}
    }
\onehalfspacing

\pagestyle{fancy}
\renewcommand{\sectionmark}[1]{\markright{#1}}
\fancyhf{} 
\fancyhead[R]{\bfseries\thepage}
\fancyhead[LO]{\docopyright}

\newcommand{\image}[2]{
	\begin{figure}[H]
		\center{\includegraphics[height=#2pt]{../../#1} }
    \end{figure}
}

\newcommand{\dotitle}[2]{

\thispagestyle{empty}
\begin{minipage}[l]{300pt}
  \vspace*{25pt}
  \sloppy{
  \scriptsize{
    \line(6,0){0}

    \centering \docopyright

    \centering Привет! Это BOTVA ИУ6, точнее малая ее часть. 
    
    \centering Пользоваться и распространять файлы конечно же можно. Если вы нашли ошибку в файле,
    
    \centering можете исправить ее в исходном коде и подать на слияние или просто написать в issue. 

    \centering {\bf{Так же вы можете купить распечатанную версию данного файла в виде книжки.}
    
    \centering Если возникнут вопросы, пишите в комментарии под постом файла в tg.}
    
    \centering Приятного бота)

    \line(6,0){300}

    \centering \href{GitHub}{https://github.com/pluttan}

    \line(6,0){0}}
    }
\end{minipage}
\begin{minipage}[r]{150pt}
%   \raggedright\includegraphics[height=150pt]{/root/BMSTU/forMyDocs/preamb/qr1.png}
\end{minipage}

\topskip=-250pt
\vspace*{140pt}
\begin{Huge}
  \textbf{
    \begin{center}
        #1
    \end{center}
  }
  {\begin{center} 
        #2
    \end{center}}
\end{Huge}
\vspace*{300pt}
  \begin{flushright}
    Над файлом работали: \\
    \copyright
  \end{flushright}
\newpage
\pagestyle{fancy}
\renewcommand{\sectionmark}[1]{\markright{#1}}
\fancyhf{} 
\fancyhead[R]{\bfseries\thepage}
\fancyhead[LO]{\docopyright}
}
\renewcommand{\sectionmark}[1]{\markright{#1}}
\renewcommand{\sectionmark}[1]{\markright{#1}}
% \makeatletter
%      \renewcommand*\l@section{\@dottedtocline{1}{0em}{1.8em}}
%      \renewcommand*\l@subsection{\@dottedtocline{2}{1.5em}{2.0em}}
%      \renewcommand*\l@subsubsection{\@dottedtocline{3}{4.3em}{3.0em}}
% \makeatother
\newcommand{\toc}{
\newpage
\renewcommand{\contentsname}{Оглавление}
\large{\tableofcontents}
\newpage
\large
}


% It's XeLaTeX, if you can't compile comment this line 
\usepackage{fontspec}\setmainfont{Times New Roman} 
